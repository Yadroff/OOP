\documentclass[12pt]{article}

\usepackage{fullpage}
\usepackage{multicol,multirow}
\usepackage{tabularx}
\usepackage{ulem}
\usepackage[utf8]{inputenc}
\usepackage[russian]{babel}
\usepackage{minted}
\usepackage{color} %% это для отображения цвета в коде
\usepackage{listings} %% собственно, это и есть пакет listings
\lstset{ %
language=C++,                 % выбор языка для подсветки (здесь это С++)
basicstyle=\small\sffamily, % размер и начертание шрифта для подсветки кода
numbers=left,               % где поставить нумерацию строк (слева\справа)
%numberstyle=\tiny,           % размер шрифта для номеров строк
stepnumber=1,                   % размер шага между двумя номерами строк
numbersep=5pt,                % как далеко отстоят номера строк от подсвечиваемого кода
backgroundcolor=\color{white}, % цвет фона подсветки - используем \usepackage{color}
showspaces=false,            % показывать или нет пробелы специальными отступами
showstringspaces=false,      % показывать или нет пробелы в строках
showtabs=false,             % показывать или нет табуляцию в строках
frame=single,              % рисовать рамку вокруг кода
tabsize=2,                 % размер табуляции по умолчанию равен 2 пробелам
captionpos=t,              % позиция заголовка вверху [t] или внизу [b] 
breaklines=true,           % автоматически переносить строки (да\нет)
breakatwhitespace=false, % переносить строки только если есть пробел
escapeinside={\%*}{*)}   % если нужно добавить комментарии в коде
}


\begin{document}
\begin{titlepage}
\begin{center}
\textbf{МИНИСТЕРСТВО ОБРАЗОВАНИЯ И НАУКИ РОССИЙСКОЙ ФЕДЕРАЦИИ
\medskip
МОСКОВСКИЙ АВИАЦИОННЫЙ ИНСТИТУТ
(НАЦИОНАЛЬНЫЙ ИССЛЕДОВАТЕЛЬСКИЙ УНИВЕРСТИТЕТ)
\vfill\vfill
{\Huge ЛАБОРАТОРНАЯ РАБОТА №8} \\
по курсу объектно-ориентированное программирование
I семестр, 2021/22 уч. год}
\end{center}
\vfill

Студент \uline{\it {Ядров Артем Леонидович, группа М8О-208Б-20}\hfill}

Преподаватель \uline{\it {Дорохов Евгений Павлович}\hfill}

\vfill
\end{titlepage}

\subsection*{Условие}

Задание: \
Вариант 27: Очередь, вектор.\
Используя структуру данных, разработанную для лабораторной работы №5, спроектировать и
разработать аллокатор памяти для динамической структуры данных.
\
Цель построения аллокатора – минимизация вызова операции malloc. Аллокатор должен
выделять большие блоки памяти для хранения фигур и при создании новых фигур-объектов
выделять место под объекты в этой памяти.
\
Алокатор должен хранить списки использованных/свободных блоков. Для хранения списка
свободных блоков нужно применять динамическую структуру данных (контейнер 2-го уровня,
согласно варианту задания).
\
Для вызова аллокатора должны быть переопределены оператор new и delete у классов-фигур.
Нельзя использовать:
\begin{enumerate}
	\item  Стандартные контейнеры std
\end{enumerate}
Программа должна позволять:
\begin{enumerate}
	\item  Вводить произвольное количество фигур и добавлять их в контейнер;
	\item Распечатывать содержимое контейнера;
	\item Удалять фигуры из контейнера.
\end{enumerate}


\subsection*{Описание программы}

Исходный код лежит в 13 файлах:
\begin{enumerate}
\item main.cpp: основная программа, взаимодействие с пользователем посредством комманд из меню

\item tqueueitem.h:    описание класса предмета очереди
\item point.h:     описание класса точки
\item tqueue.h:  описание класса очереди
\item rectangle.h: описание класса прямоугольника, наследующегося от figures
\item point.cpp:     реализация класса точки
\item tqueue.inl:  реализация класса очереди
\item rectangle.cpp: реализация класса прямоугольника
\item tqueueitem.inl:    реализация класса предмета очереди
\item vector.h : описание и реализация класса вектора
\item tallocationblok.h : описание класса аллокатора
\item tallocationblok.cpp : реализация класса аллокатора

\end{enumerate}

\subsection*{Дневник отладки}


\subsection*{Недочёты}


\subsection*{Выводы}

Я закрепил навыки работы с шаблонами классов и научился строить итераторы для динамических структур данных.


\vfill

\subsection*{Исходный код}

{\Huge tqueue.h}
\inputminted{C++}{tqueue.h}
    \pagebreak

{\Huge tqueue.cpp}
\inputminted{C++}{tqueue.inl}
\pagebreak

{\Huge tqueueitem.h}
\inputminted{C++}{tqueueitem.h}
\pagebreak

{\Huge tqueueitem.cpp}
\inputminted{C++}{tqueueitem.inl}
\pagebreak
    
{\Huge point.h}
\inputminted{C++}{point.h}
    \pagebreak

{\Huge point.cpp}
\inputminted{C++}{point.cpp}
    \pagebreak

{\Huge rectangle.h}
\inputminted{C++}{rectangle.h}
\pagebreak

{\Huge rectangle.cpp}
\inputminted{C++}{rectangle.cpp}
\pagebreak

{\Huge vector.h}
\inputminted{C++}{vector.h}
\pagebreak

{\Huge tallocationblock.h}
\inputminted{C++}{tallocationblock.h}
\pagebreak

{\Huge tallocationblock.cpp}
\inputminted{C++}{tallocationblock.cpp}
\pagebreak

{\Huge main.cpp}
\inputminted{C++}{main.cpp}
    \pagebreak
    
\end{document}
